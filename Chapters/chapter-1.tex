\color{tudelft-cyan}
\chapter{Introduction}
\color{black}


\label{intro}

%\addcontentsline{toc}{section}{Introduction} % Adds this section to the table of contents

The development of quantum mechanics in the beginning of the 20th century has had a major impact on society, as it revealed phenomena such as wave-particle duality, energy quantization and entanglement that were not in line with the current weltanschauung. [find citation] Especially the notion of entanglement has long been debated due to a paper published by Einstein, Podolsky and Rosen in 1935. They stated, by means of a thought experiment, that the theory of quantum mechanics was not complete as it does not allow local realism. \cite{einstein1935can} In order to overcome this problem, it was argued that `` hidden" variables determine the precise behavior of any individual system. \cite{bohm1952suggested,bohm1952suggested2}

However, in 1960 it was shown mathematically that any hidden variable theory is incompatible with the statistical predictions of quantum mechanics. \cite{bell1964on} This mathematical description led to the development of the CHSH-inequality which enabled physicist to design experiments that made a decisive test between quantum mechanics and local hidden-variable theories possible. \cite{clauser1969proposed}

The first experimental evidence of non-locality in quantum mechanics was given in 1982 by Aspect et. al. \cite{ aspect1982experimental, aspect1982experimental2} The system they used consisted of a pair of photons, in polarization entangled, traveling in different direction from their origin, a calcium-40 atom. By measuring the polarization of both photons in different bases, the CHSH-inequality was violated up to 83\% of its maximal predicted value. Later, many more experiments that violated the CHSH-inequality followed using a variety of systems such as photons entangled in orbital angular momentum \cite{mair2001entanglement}, photons entangled in multiple degrees of freedom \cite{barreiro2005generation}, trapped atoms \cite{rowe2001experimental}, hybrid atom-photon systems \cite{blinov2004observation} and Jospehson phase superconducting qubits. \cite{ansmann2009violation}

Unfortunately, none of these experiments were completely conclusive due to either the locality or the detection loophole. The first one mainly affects atom-based experiment due to limited spatial separation whereas the latter one mainly affects photon-based experiments because of limited photon-detection efficiency. 

Nitrogen-vacancies in diamond have recently emerged as candidate for a successful loophole-free Bell test. Firstly, because the detection loophole should not affect the system as presented recently by Pfaff et al. \cite{pfaff2014unconditional} Secondly, because the recent development of solid immersion lenses in combination with a new anti-reflection coating has increased the collection efficiency of emitted photons, the readout fidelity and the entanglement visibility even more. Finally, the construction of two laboratories separated by 1.3 km connected through an optical fiber enables the experiment necessary to close the locality loophole.  

Here we show first hurdles that are taken towards reaching a loophole free Bell test using NV-centers in diamonds. We show that by optimizing the splices between couplers and fibers an attenuation of ..... \% of the predicted value is reached. Apart from that we demonstrate that optimizing the bases in which the measurements are carried out with respect to the visibility and fake entanglement probability yields a significant improvement.


\color{tudelft-cyan}
\section{Document Structure}
\color{black}

\color{tudelft-cyan}
\subsection{Document Structure}
\color{black}

\subsubsection{\textbackslash subsubsection\{\ldots\}}



